%%%%%%%%%%%%%%%%%%%%%%%%%%%%%%%%%%%%%%%%%%%%%%%%%%%%%%%%%%%%%%%%%%%%%%%%
% Plantilla TFG/TFM
% Escuela Politécnica Superior de la Universidad de Alicante
% Realizado por: Jose Manuel Requena Plens
% Contacto: info@jmrplens.com / Telegram:@jmrplens
%%%%%%%%%%%%%%%%%%%%%%%%%%%%%%%%%%%%%%%%%%%%%%%%%%%%%%%%%%%%%%%%%%%%%%%%

\chapter{Marco Teórico}
\label{marcoteorico}


Recientes investigaciones (MIT) proponen arquitecturas más complejas técnicas para crear mapas de riesgos de accidentes, tomando como input imágenes tomadas por satélite, segmentación de carreteras, información GPS de los vehículos y datos de accidentes con la finalidad de

que pueden ser monitorizados en tiempo real para ayudar a mitigarlos.

