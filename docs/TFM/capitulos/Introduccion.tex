%%%%%%%%%%%%%%%%%%%%%%%%%%%%%%%%%%%%%%%%%%%%%%%%%%%%%%%%%%%%%%%%%%%%%%%%
% Plantilla TFG/TFM
% Escuela Politécnica Superior de la Universidad de Alicante
% Realizado por: Jose Manuel Requena Plens
% Contacto: info@jmrplens.com / Telegram:@jmrplens
%%%%%%%%%%%%%%%%%%%%%%%%%%%%%%%%%%%%%%%%%%%%%%%%%%%%%%%%%%%%%%%%%%%%%%%%

\chapter{Introducción}

\section {Motivación}


	Los accidentes de tráfico son uno de los principales problemas de Salud Pública en nuestros días. La multicausalidad, la variedad de las fuentes de datos y la poca cantidad de análisis específicos, apuntan a una gran complejidad en su tratamiento. Mas de 3.500 personas mueren, en las carreteras, diariamente en todo el mundo. Esto, significa casi 1,3 millones de muertes y aproximadamente 50 millones de lesiones cada año. Por lo tanto, la predicción de la gravedad de los accidentes de tráfico es un componente muy importante que se debe tener en cuenta, adoptando cualquier mejora en la predicción de la gravedad de los mismos. Además, el impacto económico y social asociado con los accidentes de tráfico lleva a las administraciones a buscar de manera activa mejoras.

	En el campo de la investigación sobre seguridad vial, el desarrollo de metodologías fiables para predecir y clasificar el nivel de gravedad, de los accidentes de tráfico, en función de diversas variables es un componente clave. La creciente disponibilidad de datos a gran escala de varias fuentes, incluidos los vehículos conectados y autónomos exige una comprensión profunda de las relaciones causales entre la seguridad y factores asociados con la ayuda de nuevas metodologías. Aprovechar bien esta información es imprescindible para desarrollar nuevos sistemas optimizados de prevención y reacción, como por ejemplo el despliegue inteligente de ambulancias en función de la gravedad de las víctimas tras producirse un accidente.

	En las últimas dos décadas, los rápidos desarrollos en los métodos de aprendizaje automático (ML), su regresión precisa y el rendimiento de clasificación han atraído la atención de los investigadores, provocando un número creciente de aplicaciones de métodos ML en la investigación de la gravedad de los accidentes. Aunque los métodos estadísticos tradicionales tienen formas funcionales rigurosas y claramente definidas, los métodos ML tienen una gran flexibilidad, requieren poca o ninguna suposición previa con respecto a los datos de gravedad de choques y son capaces de manejar valores perdidos, ruidos y valores atípicos.

	A pesar de que la utilización de técnicas de inteligencia artificial (ML) permite identificar las relaciones ocultas entre los factores de accidentes, el principal inconveniente de estos modelos es que pocos estudios modelos de aprendizaje profundo para mejorar el rendimiento, lo que se traduce en un bajo rendimiento en  accidentes graves. Además, como los conjuntos de datos de accidentes de tráfico existentes, están muy desequilibrados, es difícil mejorar el rendimiento de las clases menos comunes como en el de los accidentes graves. Basándose en lo expuesto anteriormente, en este trabajo se plantea estudiar y formalizar un modelo de aprendizaje profundo para predicción de la gravedad de los accidentes de tráfico. Debido a las características de los datos y para preparar la entrada al modelo de aprendizaje profundo, se ha utilizado una técnica para transformar números en matrices bidimensionales \cite{TASPCNN}. Esta técnica consiste en convertir un vector de características del accidente en una matriz de características, teniendo en cuenta que los vectores de características son las variables de los datos del accidente. Los resultados obtenidos se han comparado con otros resultados de modelos de aprendizaje automático con la finalidad de extender las conclusiones.

\section {Objetivos}

	\subsection{Objetivos Generales}

		El objetivo principal de este trabajo final de Máster es desarrollar un sistema predictivo de la gravedad de accidentes de tráfico utilizando características que se pueden identificar en los lugares del accidente, como el sexo del conductor, el tipo de vehículo, el entorno o los atributos de la carretera.

	\subsection{Objetivos Específicos}

		Para alcanzar este objetivo principal, se plantean los siguientes objetivos específicos:

		\begin{enumerate}
		    \item Utilización de técnicas para la transformación de características cualitativas de los accidentes de tráfico en matrices numéricas.
		    \item Utilización de algoritmos de aprendizaje automático basado en árboles de decisiones (XGBoost) para inferir pesos de las características de accidentes de tráfico.
		    \item Utilizar técnicas de algoritmia evolutiva para optimización de parámetros de entrada del algoritmo XGBoost.
		    \item Desarrollar un enfoque basado en el aprendizaje profundo, redes convolucionales, para predecir la gravedad de los accidentes de tráfico.
		    \item Comparar los resultados de los modelos de aprendizaje profundo con otros modelos utilizando indicadores de rendimiento.
		\end{enumerate}

