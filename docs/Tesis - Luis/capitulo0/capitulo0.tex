\chapter*{Abstract}

%\addtocounter{chapter}{0}

%\textcolor{blue}{\textbf{Luis: en principio OK, pero hay que repasarlo, me he liado en algún punto, la intuición más o menos creo que sería esta.}}\\

%\textcolor{purple}{\textbf{Jose: acabado.}}\\

%\textcolor{green}{\textbf{Manu: acabado, aunque hay 3 líneas que pasan a la siguiente página y sería interesante que el resumen se quede en una sola página....}}\\

%\textcolor{orange}{\textbf{Luis: } lo miro mañana.}

En esta tesis se presenta un nuevo modelo general que predice la necesidad de asistencia médica en accidentes de tráfico, independientemente de la ciudad en que se produzca, en base a información sobre la descripción del accidente. Conocer la gravedad del accidente una vez se produce es de vital importancia, ya que permite asignar recursos médicos de forma eficiente una vez se conocen las características del mismo, permitiendo evitar así consecuencias más graves en los afectados a corto, y largo plazo al disponer de asistencia médica en un tiempo acorde a la gravedad del mismo. Con el objetivo de implementar un modelo general que pueda ser aplicado en distintas regiones independientemente de los datos disponibles, debido principalmente a las limitaciones socioeconómicas de la región, se presenta un modelo generalizable que permite adaptar cualquier conjunto de datos recibido a la entrada de este nuevo modelo clasificador.

Los modelos de clasificación existentes actualmente para este caso de uso tienen, como principal desventaja, que las características que requieren para sus predicciones deben ser las mismas respecto a los datos con los que se han entrenado. Es decir, requiere de un desarrollo específico para cada dataset en la que se quisiese aplicar, ya que cada una de estos, por la naturaleza socioeconómica de las poblaciones sobre los que se aplica, puede no recoger ciertos datos que sí están presentes en otras. 

El modelo diseñado en esta tesis permite solventar este problema mediante un enfoque basado en la categorización de las características de los accidentes, donde en función de la naturaleza de cada dato disponible estos puedan ser asignados a categorías que engloban información a un nivel más alto, permitiendo así que el nuevo modelo propuesto sea independiente a los datos que estén disponibles en la región.

%De esta forma, si se pretendiese diseñar un modelo general, con independencia de la información disponible en cada caso, no sería posible, y requeriría de un desarrollo específico para cada población en la que se quisiese aplicar, ya que cada una de estas, por la naturaleza socioeconómica de las poblaciones, puede no recoger ciertos datos que sí están presentes en otras. 

Para validar este enfoque se compararán los resultados de este nuevo modelo con otros seis modelos del estado del arte que han sido aplicados históricamente para la predicción de la necesidad de asistencia médica en accidentes a lo largo de regiones distintas en distintos países, donde la información disponible en cada uno de estos conjuntos de datos es distinta por la naturaleza socioeconómica de regiones. Además se utilizarán técnicas para evaluar la robustez del nuevo modelo mediante prubas de estrés, donde para cada uno de los conjuntos de datos se irán eliminando características de mayor y menor importancia y se reevaluarán estos resultados commparándolos con a los modelos del estado del arte.


