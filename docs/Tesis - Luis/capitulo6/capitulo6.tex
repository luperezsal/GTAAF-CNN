
\chapter{Conclusiones}


En esta tesis se ha propuesto un nuevo modelo que evalúa la necesidad de asistencia médica en los accidentes de tráfico. Esta funcionalidad es extremadamente importante para priorizar la asignación de recursos médicos una vez se conocen las características del accidente, de tal forma que se puedan minimizar las consecuencias físicas a corto y largo plazo de los afectados. Para ello se ha propuesto un modelo que transforma las características que describen los accidentes, mediante categorizaciones, en datos matriciales para alimentar a nuestro modelo convolucional GTAAF. Como se ha demostrado en su evaluación, los resultados no solo mejoran ampliamente a los modelos del estado del arte (con valores de hasta el 13.8\%), sino que la categorización propuesta ha demostrado robustez respecto a la interpretación de las características de forma individual de los demás modelos. Además, nuestro modelo ha mostrado un gran rendimiento en distintos contextos, concretamente en  distintos datasets de 8 poblaciones de diferentes densidades de población.\\

Además, con el objetivo de proponer un modelo general que pueda ser aplicado a nuevas poblaciones que no dispongan de la misma información que los datasets presentados en este artículo (debido principalmente a la dificultad inherente de recogida de datos específicos, como controles de alcohol y drogas, u otras características cuya obtención esté relacionada con la condición económica de la población), se ha analizado la robustez del modelo eliminando características, excluyendo aquellas de mayor y menor impacto que han resultado del \textit{XGBoost} optimizado mediante el algoritmo genético, obteniendo resultados incluso mejores en nuestro modelo. Esto hace indicar la sensibilidad que tiene respecto a estos casos extremos.

Como principal medida medida como trabajo a futuro se propone aplicar este modelo a otros conjuntos de datos de accidentes para evaluar la utilidad que ofrece en comparación con otros modelos, por otra parte se debe analizar cómo reducir la sensibilidad del modelo GTAAF ante características extremas.


